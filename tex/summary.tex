% !TEX root = ../thesis.tex

\begin{summary}
本文对内存完整性保护方案做了系统的分析,就内存完整性保护的大小,树中节点数据结构,危险模型与安全性分析做了仔细地探讨。通过对已有的内存完整性保护的研究发现,虽然已有的工作能够理论上实现
对GB级别的内存数据做完整性检查,但是任然不适合云场景中的TB级别的内存,以及安全内存的可扩展性。在本文中,我们提出了一种新颖的完整性保护数据结构:哈希森林以及可挂载完整性保护树。其中挂载的单位是子树,在MMT中子树还会处于三个状态:活跃,不活跃与未分配
并且针对子树的概念提出了两个新的动态操作:挂载(Mount)与卸载(Unmount)。本文通过这两个新增的完整性保护树的动态原语,实现了可扩展的内存完整性保护,动态的安全内存分配以及针对离散内存的保护。同时结合了内存访问的规律,
提出了三级counter的树节点架构,针对冷热counter做了进一步的优化,从而实现了在不影响安全性的前提下,增加了额树节点的扇出,扩大了内存保护的范围。

在论文中,实现了基于MMT完整性保护树,以及哈希森立的内存保护原型。通过在内存控制器中增加三个部件:Secure Bitmap;Mount Table;Root-of-root来实现子树的挂载,内存的完整性检查等操作。同时对内存中的布局就做进一步的细化,将内存区域分成了:非安全内存,
安全内存,子树区域与MMT元数据区域,为了保护MMT元数据区域内的数据不被攻击者篡改,我们才用了RISCV架构的M mode中monitor对该区域进行管理,同时采用pmp寄存器对MMT元素据区域进行隔离,保证特权软件无法篡改其中的子树的根节点。

在论文中,对针对内存的攻击进行了详细的分析与归类,并且通过密码学的方式证明了对内存完整性保护的可行性,同时也分析了不同数据结构可能带来的安全隐患。论文中详细分析MMT与VAULT, BMT等相关工作的安全性,虽然MMT采用了更加激进的counter组织方式,但是得益于对内存
访问pattern的观察,MMT会比VAULT有更强的安全性(更低的概率实现重放攻击)。之后的测试也表明,MMT所采用的树的数据结构并不会带来更多的运行时开销。

在实现中,对于树的配置:MMT子树的层数为3,保护的内存为4M,根树的层数时3,保护的内存为2M;SoC上的开销:Secure Bitmap大小为16K,Mount Table大小为512B;内存开销:MMT元数据区域大小约为2M。在极少的完整性保护树的层数下(<SGX的6层),极少了SoC空间开销的情况下(16Kb空间),我们实现了
对512G的内存数据的完整性保护,同时可有SoC保护的内存为128M。针对不同的情况,可以增加树的层数,以及Mount Table的容量,实现更多内存的完整性保护。

论文基于gem5模拟器实现了VAULT,SGX,MMT等多种内存完整性保护方案。首先我们测试了传统的交换页的机制对性能的影响,测试结果表明,交换页对占总运行时间的百分之40以上;其次我们测试在极限场景下,传统完整性保护方案的开销(7\sim8倍的开销);最后我们基于SPECCPU对MMT和其他完整性保护方案进行了对比,
MMT均取得了最好的性能且Mount操作带来的额外开销小于1\%,保证了内存完整性保护的可扩展性。
\end{summary}
